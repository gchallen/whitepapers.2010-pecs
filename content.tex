Throughout the history of computing, testbeds have played major roles in
advancing research. Multiple areas have reached a point at which small,
limited, single-experiment instrumentation failed to provide the power, scale
and realism needed to make progress. At each critical juncture, a group of
scientists built shared infrastructure that helped carry the field forward.

We believe that smartphone research has reached this critical moment. Smartphones are one of the most rapidly-adopted technologies in human history, and conference proceedings are bulging with papers attempting to harness the power of this emerging technology. As the smartphones change the way that we communicate and interact, scientists from a broad range of disciplines are anxious to study its effects and unlock its potential. But all of these efforts are stalled by the lack of a testbed remotely approximating the reality we are witnessing.

%\textbf{\textsc{Broader Impact:}} \PhoneLab{} will be a publicly-available
%open-access testbed. Considering the research impact of other similar
%instruments --- EmuLab on local-area networking, PlanetLab on wide-area
%networking, and MoteLab on sensor networking --- we anticipate that
%\emph{\PhoneLab{} will accelerate phone cloud research.} Given the sensing
%and inference capabilities of the phone cloud, we expect \PhoneLab{} to
%engage scientists from a broad range of disciplines. Engineers, sociologists,
%psychologists, public health researchers, and many other non-CSE users will
%find uses for the testbed, and \PhoneLab{} is designed to enable their
%research as well.

We envision a publicly-available smartphone testbed called \emph{PhoneLab} that enables smartphone operating system and mobile application research in a realistic environment at a scale not previously possible. PhoneLab will consist of a large number of (say 1,000) reprogrammable Android devices used by students, staff, and faculty in a university campus, and will be  supported by a back-end data center. PhoneLab will provide \emph{power}, allowing the modification of smartphone software above and below the OS-application interface, while simplifying instrumentation and data collection to facilitate efficient experimentation. PhoneLab will provide \emph{scale}, allowing researchers access to an order of magnitude more participants than typically used by smartphone studies. By minimizing experimental disturbance, PhoneLab will provide \emph{realism}, ensuring that participants use their smartphones as they would normally. We believe that PhoneLab will accelerate research on phone cloud applications, networking, infrastructure, and system software, providing a standardized environment where experiments can be validated and competing approaches compared.

We believe that our vision for PhoneLab can enable the next-generation of mobile cloud computing research by providing the above-mentioned power, scale, and realism. Considering the research impact of other similar testbeds --- EmuLab and PlanetLab on networking and distributed systems as well as MoteLab on sensor networking --- we anticipate that PhoneLab can accelerate mobile cloud computing research.

PhoneLab will consist of three layers --- phone, infrastructure, and
interface --- and we briefly describe how they work together to provide its
core functionality.

\emph{Phone:} We will distribute 1,000 next-generation Android
phones to PhoneLab participants. We have chosen the Android platform
because it is the only open-source mobile device operating system
currently available.  Access to the operating system layer allows
PhoneLab experimenters to perform studies that cannot be performed via
the Apple Store or Android Market.

\emph{Infrastructure:} The infrastructure component of
PhoneLab will consist of 4G base stations and server infrastructure installed
on campus. 4G access points allow us to provide data access to our
participants for free --- a significant participation incentive --- while
producing a next-generation smartphone environment for experimentation. A
back-end data center will collect and store experimental data for further
analysis, while also allowing users to run code interacting with software
deployed on the phones.

\emph{Interface:} The access layer will consist of interfaces used
by both users and participants. Users will use their interface to design,
schedule, and monitor their PhoneLab experiments. Participants will use
their interface to track the experiments they are part of and monitor the
information being collected about them.



\section{Potential Advances} There are four research areas
PhoneLab will help advance. 
 
%For each, we both list some of the pressing research issues in that area and %describe an example experiment that PhoneLab will enable.

\emph{Applications:} PhoneLab will enable the community to attack emerging research issues in a wide array of application domains such as crowd-sourcing, social networking, user interaction, environmental sensing, and epidemiology studies. 

\iffalse
One of the authors, Murat Demirbas, is currently doing a project in this category, that uses smartphones to collect time-activity data for pollutant exposure estimation. PhoneLab would provide a realistic experimentation platform for this research.
\fi

\emph{Infrastructure:} PhoneLab will also enable research into infrastructural
issues such as environmental interaction, task distribution, and new wireless
technologies. In particular, we believe PhoneLab can be instrumental in designing and developing of the cloud-backed smartphone infrastructure, which uses nearby computers or data centers to increase the computational and storage power of the smartphone.

%For example, Byung-Gon Chun at Intel Labs Berkeley expressed
%his interest in using PhoneLab for his CloudClone project that
%``uses nearby computers or data centers to speed up your smart phone
%applications''. PhoneLab would provide an ideal environment for testing
%his approach.

\emph{Networking:} PhoneLab will provide a great platform for networking
research such as multi-radio issues, delay-tolerance, and peer-to-peer.
Many research project investigate ways to utilize multiple communication
technologies --- voice, SMS, WiMax, WiFi, Bluetooth --- each with its own
capabilities and limitations. PhoneLab is the perfect testbed to experiment
with an integrated networking layer that attempts to divide traffic between
multiple radios to improve performance.

\emph{Operating Systems:} PhoneLab will allow researchers to modify OS
components, and help investigate issues such as mobile operating system
design, distributed systems, energy management, and fault-tolerance.

%One of the authors, Geoffrey Challen, is investigating more efficient energy
%management schemes, and PhoneLab would allow this work to instrument
%devices and study energy consumption patterns over a wide user base.

\section{Investigators}
%Building PhoneLab requires expertise in smartphones, distributed systems,
%cloud computing, and testbed design. Our team has complementary expertise in
%these areas, with \emph{each team member having built large-scale instruments
%early in their research careers}.
We highlight each author's previous work most relevant to the topic of this
white paper, \ie, small devices, cloud computing, and testbed design.
\emph{Geoffrey Challen} developed and
%maintained \href{http://motelab.eecs.harvard.edu}{MoteLab}~\cite{motelab-ipsn},
maintained MoteLab, the first wireless
sensor network testbed consisting of 200 sensor nodes and supporting over
700~users. \emph{Murat Demirbas} helped develop and deploy the ``Line In
%The Sand'' 100-node wireless sensor network~\cite{lites} for
The Sand'' 100-node wireless sensor network for
detection, classification, and tracking, which led to the 1,000-node
%``ExScal'' network~\cite{exscal}. 
``ExScal'' network. \emph{Steven Ko} helped design the
%\href{http://opencirrus.org}{HP/Intel/Yahoo! OpenCirrus$^{TM}$ Cloud
HP/Intel/Yahoo! OpenCirrus$^{TM}$ Cloud
%Computing Testbed}~\cite{opencirrus-ieee},
Computing Testbed, a federated multi-datacenter testbed spanning over
14 institutions in US, Europe, and Asia and including more than one
thousand servers. \emph{Tevfik Kosar} designed and developed both the
%Stork distributed data scheduling system~\cite{stork}
Stork distributed data scheduling system currently used by institutions
%worldwide and the PetaShare distributed storage network~\cite{petashare}
worldwide and the PetaShare distributed storage network that manages more
than 700 Terabytes of storage located across nine university campuses in
Louisiana.


\end{document}
