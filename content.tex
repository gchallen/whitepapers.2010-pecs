\pagestyle{empty}
Throughout the history of computing, testbeds have played major roles in
advancing research. Multiple areas have reached a point at which small,
limited, single-experiment instrumentation failed to provide the power, scale
and realism needed to make progress. \textbf{We believe smartphone research
has reached this critical juncture.} Large-scale shared infrastructure is now
needed.

\section{Vision}

Smartphones are one of the most rapidly-adopted technologies in history.
Conference proceedings are bulging with papers attempting to harness their
power, and as smartphones change the way that we communicate and interact,
scientists from a broad range of disciplines are anxious to study their
effects. But these efforts are stalled by the lack of a testbed remotely
approximating the reality we are witnessing.

We envision a publicly-available smartphone testbed called {\scshape
PhoneLab} that enables smartphone operating system and mobile application
research in a realistic environment at a scale not previously possible.
{\scshape PhoneLab} will consist of a large number ---1,000 or more --- of
reprogrammable Android devices used by students and staff at a university
campus. {\scshape PhoneLab} will provide:

\begin{itemize}

\item \textbf{Power:} allowing the modification of smartphone software above
and below the OS-application interface, while simplifying instrumentation and
data collection to facilitate efficient experimentation.

\item \textbf{Scale:} providing access to an order of magnitude more
participants than typically used by smartphone studies.

\item \textbf{Realism:} by minimizing experimental disturbance and allowing
participants to use their smartphones naturally.

\end{itemize}

Compared to the Apple AppStore or Android Market, {\scshape PhoneLab}
provides a controlled, yet realistic environment with a large number of
participants in one location. We anticipate {\scshape PhoneLab} supporting
experiments on emerging topics such as social networking, cyber-physical
systems, smart environments, smartphone sensing, and crowd-sourcing. In
addition, access to the operating system allows PhoneLab experimenters to
perform studies that cannot be performed by installing applications. We
believe that {\scshape PhoneLab} will accelerate research on
smartphone-related areas by providing a standardized environment where
experiments can be validated and competing approaches compared.

Considering the research impact of other similar testbeds --- EmuLab and
PlanetLab on networking and distributed systems, MoteLab on sensor networking
--- we expect {\scshape PhoneLab} to accelerate mobile cloud computing
research. To bring this vision to fruition, we are designing an initial
{\scshape PhoneLab} prototype.

\section{Potential Advances} To demonstrate the kinds of research that
{\scshape PhoneLab} will enable, we identify four main research areas the
testbed will advance. 

\begin{itemize}

\item \textbf{Applications:} {\scshape PhoneLab} will enable the community to
tackle emerging research issues in a wide array of application domains such
as crowd-sourcing, social networking, user interaction, environmental
sensing, and epidemiology studies. For example, {\scshape PhoneLab} would
provide an experimentation platform for collecting sensor data such as
time-activity traces for pollutant exposure estimation. {\scshape PhoneLab}
could also assist the development of more realistic human mobility or social
network models.

\item \textbf{Infrastructure:} {\scshape PhoneLab} will support research into
infrastructure aspects such as environmental interaction, task distribution,
and new wireless technologies. For example, we believe that {\scshape
PhoneLab} can be instrumental in designing and developing a cloud-backed
smartphone infrastructure using nearby computers or data centers. There are
many research questions in this area concerning off-loading computation,
hierarchical storage designs that extend the limited smartphone storage to
data centers, and privacy and security issues arising from interaction with
data centers.

\item \textbf{Networking:} {\scshape PhoneLab} will provide an 
environment for networking research such as multi-radio fusion,
delay-tolerant protocols, and peer-to-peer interaction. Many research
projects investigate ways to utilize multiple communication technologies ---
voice, SMS, WiMax, WiFi, Bluetooth --- each with its own capabilities and
limitations. {\scshape PhoneLab} would allow the deployment of an integrated
networking layer that attempts to divide traffic between multiple radios to
improve performance. Moreover, while 4G is slowly being adopted nationwide,
there are few studies reporting 4G behavior in the wild.

\item \textbf{Operating Systems:} {\scshape PhoneLab} will allow researchers
to modify OS components, and help investigate issues such as mobile operating
system design, distributed systems, energy management, and fault-tolerance.
Operating systems modifications can be tested and compared on {\scshape
PhoneLab} as experimenters will have access to the same group of participants
with the same usage patterns over time. For example, experimenters can run a
number of different energy management approaches over the span of a few weeks
and compare the results.

\end{itemize}

\section{Investigators}

We briefly highlight each author's previous work most relevant to the topic
of this white paper:

\textbf{Geoffrey Challen} developed and maintained MoteLab, the first
wireless sensor network testbed consisting of 200 sensor nodes and supporting
over 700~users. \textbf{Murat Demirbas} helped develop and deploy the ``Line
In The Sand'' 100-node wireless sensor network for detection, classification,
and tracking, which led to the 1,000-node ``ExScal'' network. \textbf{Steven
Ko} helped design the HP/Intel/Yahoo! OpenCirrus\texttrademark \space Cloud Computing
Testbed, a federated multi-datacenter testbed spanning over 14 institutions
in US, Europe, and Asia and including more than one thousand servers.
\textbf{Tevfik Kosar} designed and developed both the Stork distributed data
scheduling system currently used by institutions worldwide and the PetaShare
distributed storage network that manages more than 700 Terabytes of storage
located across nine university campuses in Louisiana.

\end{document}
